\documentclass{article}
\usepackage{float,amsmath}
\usepackage{graphicx}
\usepackage{color}
\usepackage[letterpaper,margin=1in]{geometry}
\usepackage{hyperref}

%\setlength{\textwidth}{6.5in}

\begin{document}

\author{David DeBoer}
\title{Array Configuration Management Database}
\maketitle

\section{Introduction}
The HERA array/part configuration management database is a set of five tables within the larger hera\_mc database, which is maintained on-site in the Karoo.  The tables are detailed in the Appendix, 
but they are:  
\begin{tabular}{l l l}
         {\bf psql table} & {\bf python file}  &  {\bf class name} \\
	geo\_location 	& hera\_mc/geo\_location.py & GeoLocation \\
	station\_meta 	& hera\_mc/geo\_location.py & StationMeta \\
	parts\_paper 	& hera\_mc/part\_connect.py & Parts \\
	part\_info 	         & hera\_mc/part\_connect.py & PartInfo \\
	connections 	& hera\_mc/part\_connect.py & Connections \\
\end{tabular}

Software is contained in the repository https://github.com/HERA-Team/hera\_mc.

The databases are structured primarily around {\em parts} and {\em connections}.  {\em Parts} are meant to be single items that, in theory at least, are a thing than can be replaced as a unit.  
{\em Connections} define {\em ports} on a given {\em part} and connect two {\em ports} together.  All {\em parts} and {\em connections} are timed in that they have a start and stop time of operation.  If stop is {\tt None}, then it is active (it is given a date in the relatively far future).  There are currently two special parts (one at each end of the signal chain):
\begin{itemize}
	\item {\bf geo\_location}: a geo\_located part (``station'') is a named location with UTM coordinates which also has an entry in the geo\_location table;
	\item {\bf f\_engine}:  each input to the f\_engine ROACH-2 is currently listed as a separate part.  Probably it should be a 32-input part, but given timing will wait for the new architecture.
\end{itemize}

Parts are hooked together via connections of their ports.   For parts in the PAPER era, the signal chain hook-up is given below and shown in Fig. \ref{fig:hookup}.
\begin{itemize}\setlength\itemsep{-.3em}
	\item {\bf Station}: geo\_located position.  See prefixes in table station\_meta ({\em e.g.} {\tt HH} for herahex)
	\item {\bf Antenna}:  collecting element ({\em this used as the correlator number,} {\tt \{C\}}):  {\tt{\bf A}\{C\}}
	\item {\bf Feed}:  element feed.  {\tt {\bf FDP}\{\#\}} are PAPER feeds, {\tt {\bf FDA}\{\#\}} are design A HERA feeds.
	\item {\bf Frontend}:  lna, etc at feed.  {\tt {\bf FEA}\{\#\}} is design A (75 $\Omega$)
	\item {\bf Cable\_feed75}:  cable from front-end to receiverator: {\tt {\bf C7F}\{\#\}}
	\item {\bf Cable\_receiverin}:  cable inside {\tt R}$^{th}$ receiverator to receiver: {\tt {\bf RI}\{R\}\{"A"/"B"\}\{\#\}\{"E"/"N"\}}
	\item {\bf Receiver}:  receiver module in receiverator: {\tt {\bf RCVR}\{SN\}}
	\item {\bf Cable\_receiverout}:  cable inside receiverator from receiver {\tt {\bf RO}\{R\}\{"A"/"B"\}\{\#\}\{"E"/"N"\}}
	\item {\bf Cable\_receiverator}:  cable from receiverator to container {\tt {\bf CR}\{R\}\{"A"/"B"\}\{\#\}\{"E"/"N"\}}
	\item {\bf Cable\_container}:  cable inside container from {\tt\{P\}}late/{\tt\{Row\}}/{\tt\{Col\}}umn {\tt {\bf CC}\{P\}{\bf R}\{Row\}{\bf C}\{Col\}}
	\item {\bf F\_engine}:  {\tt\{R2\}} Roach-2 input {\tt {\bf DF}\{R2\}\{"A"-"H"\}\{\#\}}
\end{itemize}
Summarizing with prefixes, we have {\tt HH-A-FD-FE-C7F-RI-RCVR-RO-CR-CC-DF}.

In the hook-up mapping listing (using the -m flag), the Antenna number (without the {\tt A}) is also displayed under the station just to help the user find the antenna number that the correlator cares about.  It maps antenna to position to correlator input.

\begin{figure}[H]
\includegraphics[width=0.8\textwidth]{hookup.pdf}
\centering
\caption{Block diagram of hookup.  Filled boxes are cables, blue objects are in the receiverator and green objects are in the container.
The line labels indicate the port names/connections.}
\label{fig:hookup}
\end{figure}

\section{Use Cases}
Three use cases are identified:
\begin{enumerate}\setlength\itemsep{-.3em}
	\item In-the-field
	\item Remote user
	\item Bulk database
\end{enumerate}

\subsection{In-the-field updates}
This is the ``operational'' mode, where inputs to the on-site database are done via scripts (command-line and/or web), which update the database directly.  These scripts will be documented under the repository software.

\subsection{Remote user}
Remote use is the case when a remote user initializes and uses the tables outside of the main database.  This assumes that postgresql etc is configured on the remote computer.  Note that this mode supports both offline use (essentially steps 3-5 below, which are in bold) and development (probably all steps).  The process is as follows: 
\begin{enumerate}\setlength\itemsep{-.3em}
	\item Generate ascii files of the tables in the main database (in the Karoo) {\tt (cm\_package.py [--tables table1,table2,...)}
	\item Push to GitHub
	\item {\bf Pull down to remote computer}
	\item {\bf Install ({\tt python setup.py install})}
	\item {\bf Initialize the updated ones ({\tt cm\_initialize.py})}
	\item If you are developing, make a base version before you start ({\tt cm\_package.py --base})
\end{enumerate}

Note that generally you should leave off the {\tt --tables} option, since you should use all the tables so as to not break foreign keys (it is left there for ``expert'' users).  Steps 3-5 are bolded, since if you just want to pull down the latest and use it assuming things are up-to-date they are all that are needed (i.e., leave the ``packaging to the experts'').

This has another option of {\tt --base}, so that you can set up a base copy to which you can revert if need be by including {\tt --base} on both commands (step 6).
	
\subsection{Main database updates}
This is effectively the opposite of the remote user case, bringing an updated set of table(s) back into the main database.  The process is as follows:
\begin{enumerate}\setlength\itemsep{-.3em}
	\item  Generate ascii files of the desired tables from the remote database: 
		({\tt cm\_package.py --maindb \$key [--tables table1,table2,...]})
	\item Push to GitHub
	\item Pull down to site computer
	\item Install
	\item Update existing tables  ({\tt cm\_initialize.py --maindb \$key [--tables table1,table2,]})
\end{enumerate}

Note that {\tt \$key} is a user supplied string to just make sure you meant to do the main database update.  It can be anything, you
just need to remember it when you reload it on the main database computer qmaster.
As per above, you should generally leave off the {\tt --tables} option so as to do them all.  

\section{Tables}
As mentioned above, there are five tables in the configuration management section of the database:  (1) geo\_location, (2) station\_meta,
(3) parts\_paper, (4) part\_info, (5) connections.  The following tables summarize them with the following key:  
\begin{itemize}\setlength\itemsep{-.3em}
	\item {\bf Bold font} = primary key
	\item {\em Italics(\#)} = foreign\_key.    The \# identifies the other locations.
	\vspace{-.1in}
	\begin{itemize}\setlength\itemsep{-.3em}
		\item If {\em \underline{underlined}}, then that defines the key.
		\item If {\em {\bf bold}}, it is also a primary key.
	\end{itemize}
	\vspace{-.08in}
	\item * = NotNull entries (includes primary and foreign keys)
\end{itemize}

\begin{table}[h]
\centering
\caption{geo\_location :: geo\_location.GeoLocation}
\begin{tabular}{| l | l | l |} \hline
{\bf Column} & {\bf Type} & {\bf Description} \\ \hline
{\bf station\_name}*  & character varying(64) & Name of position - never changes \\ \hline
{\em \underline{station\_type\_name}}*(1) & character varying(64) & Type of station \\ \hline
datum & character varying(64) & UTM datum \\ \hline
tile & character varying(64) & UTM tile \\ \hline
northing & double precision & UTM coordinate \\ \hline
easting & double precision & UTM coordinate \\ \hline
elevation & double precision & Elevation \\ \hline
created\_date* & datetime & Date and time of creation. \\ \hline
\end{tabular}
\end{table}

\begin{table}[h]
\centering
\caption{geo\_location :: station\_type.StationType}
\begin{tabular}{| l | l | l |} \hline
{\bf Column} & {\bf Type} & {\bf Description} \\ \hline
station\_type\_name*(1) &  character varying(64) &  Station type name \\ \hline
prefix* & character varying(64) & 1-2 letter prefix for part station\_name \\ \hline
description & character varying(64) &  Short description \\ \hline
plot\_marker & character varying(64) & Type of matplotlib marker \\ \hline
\end{tabular}
\end{table}

\begin{table}[h]
\centering
\caption{part\_connect :: parts\_paper.Parts}
\begin{tabular}{| l | l | l |} \hline
{\bf Column} & {\bf Type} & {\bf Description} \\ \hline
 hpn*(2) & character varying(64) & HERA part number \\ \hline
hpn\_rev*(3) & character varying(32) & HPN revision letter (A-Z) \\ \hline
hptype*  &  character varying(64) & HPN part type category \\ \hline
manufacturer\_number & character varying(64) & Unique serial number for each part \\ \hline
start\_date* & timestamp no time zone & Datetime when part/rev is activated. \\ \hline
stop\_date & timestamp no time zone & Datetime when part/rev is de-activated \\ \hline
\end{tabular}
\end{table}

\begin{table}[h]
\centering
\caption{part\_connect :: part\_info.PartInfo}
\begin{tabular}{| l | l | l |} \hline
{\bf Column} & {\bf Type} & {\bf Description} \\ \hline
hpn*(2) & character varying(64) & HERA part number \\ \hline
hpn\_rev*(3) & character varying(32) & HPN revision letter (A-Z) \\ \hline
posting\_date* & timestamp no time zone & Date information was posted \\ \hline
comment* &  character varying(1024) & Comment \\ \hline
library\_file & character varying(256) &  Librarian filename (how to get it there?) \\ \hline
\end{tabular}
\end{table}

\begin{table}[h]
\centering
\caption{part\_connect :: connections.Connections}
\begin{tabular}{| l | l | l |} \hline
{\bf Column} & {\bf Type} & {\bf Description} \\ \hline
upstream\_part*(2) &  character varying(64) & Hera part number of upstream connection \\ \hline
up\_part\_rev*(3) & character varying(32) & Hera part revision of upstream connection \\ \hline
downstream\_part*(2) & character varying(64) & Hera part number of downstream connection \\ \hline
down\_part\_rev*(3) & character varying(32) & Hera part revision of downstream connection \\ \hline
upstream\_output\_port* & character varying(64) & Output port on upstream part \\ \hline
downstream\_input\_port* & character varying(64) & Input port on downstream part \\ \hline
start\_date* & timestamp with time zone & Date when connection started \\ \hline
stop\_date & timestamp no time zone & Date when connection ended \\ \hline
\end{tabular}
\end{table}

\end{document}