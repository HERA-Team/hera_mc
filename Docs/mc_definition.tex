\documentclass{article}
\usepackage{float,amsmath}
\usepackage{graphicx}
\usepackage{color}
\usepackage[letterpaper,margin=1in]{geometry}
\usepackage{hyperref}

\usepackage{outlines}
\usepackage{enumitem}
\setenumerate[1]{label=\arabic*.}
\setenumerate[2]{label=\alph*.}
\setenumerate[3]{label=\arabic*.}
\setenumerate[4]{label=\roman*.}

\newcommand{\mc}{M\&C}


\begin{document}

\author{HERA Team}
\title{HERA Monitor and Control Subsystem Definition}
\maketitle

\section{Introduction}
HERA is an international experiment to detect and characterize the Epoch of
Reionization (EOR).  The telescope is located at the South African SKA site in
the Karoo Astronomy Reserve.  This note summarizes Monitor and Control (\mc) subsystem for HERA.

Monitor and Control provides a common place for logging of metadata and messages. The \mc\ system is built around a database with a well documented table schema and a software layer to provide a simple developer framework. It will also include various online daemons for monitoring things, and both a front end web-based user interface and a command-line interface to support analysis code.

\section{Requirements}
\begin{outline}[enumerate]
	\1 Ability to fully reconstruct the historical state of the system.
	\1 All interactions between subsystems must go through or be logged by \mc.
		\2 Both subsystems in an interaction are responsible for logging communications to \mc.
		\2 Subsystems in an interaction are responsible for logging communications to \mc.
	\1 Operational metadata (e.g. temperatures, correlator bit occupancies) must be logged to \mc.
	\1 High availability (\mc\ must not limit uptime of telescope).
	\1 \mc is a provider of information about observations to end-users and must be available to them
\end{outline}

\section{Design Specification}
\begin{outline}[enumerate]
	\1 SQL database
		\2 DB Design principle: every logical sub group has a group of tables.  One adds tables to do more things. E.g. different versions of subsystems add new tables. Operations reference which tables they use.
		\2 This document (and appendices) will contain all table definitions.
		\2 Use careful dB design to avoid duplicated data, make table links/data relationships clear, use many-to-one and many-to-many links.
		\2 Transactions must be used to ensure DB integrity.
		\2 Must be mirrored in some fashion to observer locations.
	\1 At least one SW interface layer will be provided.
		\2 It�s not required to interact with \mc.
		\2 Must support relational db (i.e. multiple column primary and foreign keys) and transactions.
	\1 Hardware
		\2 LOM capabilities
		\2 Multi-teraByte mirrored disk RAID
		\2 Backup machine available on site
\end{outline}

\section{Table Definitions}
Primary keys are bold, foreign keys are italicized.

\subsection{Observations}
\textbf{\large{hera\_obs}}: This is the primary observation definition table.
\begin{center}
 \begin{tabular}{| p{2cm} | p{2cm} | p{4cm} | p{7cm} |} 
 \hline
 column & type & description & notes \\ [0.5ex]  \hline\hline
 \textbf{obsid} & long integer & start time in floor(GPS) seconds & GPS start adjusted to be within 1 second of LST to lock observations to LST for the night \\ \hline
 start\_time\_jd & double & start time in decimal JD & the actual start time to full accuracy of the start of the observation: beginning of integration of first visibility \\\hline
 stop\_time\_jd & double & stop time in decimal JD & the actual stop time to full accuracy of the end of the observation: end of integration of last visibility \\\hline
 lst\_start\_hr & double & decimal hours from start of sidereal day & provides a quick search for overlapping LSTs \\\hline
 \end{tabular}
\end{center}

\subsection{Common tables}
\textbf{\large{hera\_obs}}: Common table structure for server status info. Same columns to be used in subsystem-specific instances of this table.
\begin{center}
 \begin{tabular}{| p{3cm} | p{2cm} | p{4cm} | p{7cm} |} 
\hline
 column & type & description & notes \\ [0.5ex]  \hline\hline
 \textbf{timestamp} & time stamp & defined by \mc &  \\ \hline
 \textbf{hostname} & string & &  \\ \hline
 ip\_address & string & host IP & how should we handle multiples? \\\hline
system\_time & time stamp & on server & \\\hline
num\_cores & integer & Number of cores &  \\\hline
cpu\_load & float & percentage & defined by total load/\# cores, 5min average  \\\hline
uptime & float &  &  \\\hline
memory\_used & float & percent used &  \\\hline
memory\_size & float & &  \\\hline
disk\_space & float & percent used &  \\\hline
disk\_size & float &  &  \\\hline
network\_bandwidth & float & fractional? & can be null \\\hline
\end{tabular}
\end{center}

\end{document}
